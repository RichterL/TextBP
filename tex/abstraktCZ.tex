Cílem této práce je návrh a implementace funkční volební a hlasovací aplikace. Primárním využitím výsledné aplikace jsou volby do akademických senátů Univerzity Tomáše Bati ve Zlíně. Výsledný systém zaručuje anonymitu voličů a poskytuje jednoduchý způsob, jak maximálně zpřístupnit volby co nejvíce studentům.

Aplikace je postavena na PHP frameworku Nette. Návrh aplikace umožňuje její relativně snadné modifikace co do rozmístění mezi několik serverů i nezávislost na použitém systému řízení báze dat. Systém je zároveň možno doplnit o volitelné způsoby autentizace uživatelů.

Při návrhu bylo využito principu slepých digitálních podpisů a přímé komunikace klienta a serveru jako efektivního způsobu zajištění anonymity.