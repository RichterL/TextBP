Cílem této práce je návrh a implementace funkční volební a hlasovací aplikace. Primárním využitím jsou volby do akademických senátů na Univerzitě Tomáše Bati ve~Zlíně a hlasování na zasedáních akademických orgánů. Výsledný systém zaručuje dodržení principů elektronických voleb a poskytuje jednoduchý způsob, jak maximálně zpřístupnit volby co největšímu počtu uživatelů.

Aplikace je postavena na PHP frameworku Nette. Návrh aplikace umožňuje její relativně snadné modifikace co do rozmístění mezi několik serverů i nezávislost na~použitém systému řízení báze dat. Systém je navržen jako modulární a je možno ho dále rozšiřovat, např. o volitelné způsoby autentizace uživatelů.

Při návrhu bylo využito principu slepých digitálních podpisů a přímé komunikace klienta a serveru jako efektivního způsobu zajištění anonymity.