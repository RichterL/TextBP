Cílem této práce bylo vytvořit funkční volební aplikaci pro potřeby Univerzity Tomáše Bati ve Zlíně, kterou bude možno využít pro volby do Akademických senátů fakult i pro hlasování v rámci nich. Během vývoje byl kladen důraz na dodržení principů objektového programování jako je SRP a další, modularitu a bezpečnost. Kód je od počátku vývoje verzován na platformě GitHub \footnote{https://github.com/RichterL/ElektronickeVolby} a finální verze byla uvolněna jako Composer projekt \texttt{richterl/elektronicke-volby}. Zdrojový kód aplikace je uvolněn pod licencí GNU GPL-3.0\footnote{https://www.gnu.org/licenses/gpl-3.0.html}.

Aplikace je plně responzivní a ve všech běžně dostupných prohlížečích se zobrazuje korektně. Aplikaci je možno snadno rozšířit o nové funkce, například autentizaci pomocí Shibboleth, dvoufaktorové ověřování nebo překladové slovníky, jejichž pomocí se aplikace snadno přetransformuje ve vícejazyčnou. Zmíněná rozšíření by bylo vhodné do aplikace zahrnout v jejím dalším vývoji. Ten je díky využití verzování výrazně usnadněn i v případě spolupráce více vývojářů.

Všechny části aplikace byly průběžně testovány a optimalizovány pro co nejrychlejší chod. Podstatného zrychlení bylo dosaženo vhodným užitím AJAX požadavků, cachováním opakovaných SQL dotazů a přímým nahráváním CSV souborů do databáze místo jejich zpracování v PHP. Až stonásobného zrychlení se dosáhlo optimalizací sčítání a dešifrování hlasů, což je výpočetně náročná operace sama o sobě. Na testovacím VPS se rychlost načítání jedné stránky administračního rozhraní pohybuje okolo 500ms, pokud se načítá pouze část stránky přes AJAX, stránka se načte za cca 50ms.

Problematika souladu s nařízením GDPR byla vyřešena nejjistějším možným způsobem - zpracovávat co nejmenší možné množství osobních údajů. Jediné osobní údaje, které aplikace zpracovává o uživatelích jsou jméno, příjmení a e-mailová adresa a to výhradně po dobu nezbytně nutnou ke konání voleb. Po ukončení a zaprotokolování výsledku voleb se předpokladá, že všechna data budou ze systému smazána. Cookies se využívá pouze pro potřeby ukládání informace o session přihlášeného uživatele.

Bezpečnost celé aplikace zajišťuje framework \textit{Nette} v kombinaci se šablonovacím systémem \textit{Latte}, které ošetřují uživatelské vstupy z formulářů a URL adres dříve než se v aplikaci použijí. Ochrana proti SQL injection útokům je poskytována rozšířením \textit{dibi}, které opět veškeré proměnné před odesláním na databázový server ošetří. Bezpečnost z pohledu principů elektronických voleb (anonymita a další) poskytuje především zvolený způsob zpracování hlasů metodou slepých podpisů.