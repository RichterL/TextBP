Jak již bylo řečeno, jedná se architektonický vzor MVP. 
Jednotlivé části je potřeba chápat jako abstraktní vrstvy, nelze si pod nimi představit konkrétní (PHP) třidy.
Jedním důvodem je možné prolínání vrstev v rámci jedné třídy, tím druhým a závažnějším je pak nepochopení celého principu MVC/P architektury. Prolínání v Nette vidíme například u jednotlivých presenterů obsahujících í část zobrazovací vrstvy (metody~\phpinline{render}) 
\koment{https://zdrojak.cz/clanky/nette-framework-refactoring/?show=comments#comment-3554}
, nepochopením celého principu je pak domění mnohých (začínajících) programátorů, že Model je jeden konkrétní objekt (entita), přičemž modelová vrstva jsou nejen entity ale i business (nebo doménová) logika a dohromady tvoří tuto modelovou vrstvu. Pokud bude tento text zmiňovat \textbf{Presenter}, je tím myšlena konkrétní třída nebo skupina tříd nikoliv vrstva.

\n{3}{Třída Presenter}

Presenter přijímá objekt \phpinline{Nette\Application\Request}, který představuje HTTP požadavek a pomocí něho určí, jaké konkrétní metody je potřeba zavolat. 

Obrázek \ref{fig:zivotniCyklusPresenteru} přehledně popisuje sled volání jednotlivých metod, přičemž jsou všechny nepovinné. Vynecháním definic všech metod by došlo pouze k odeslání statického obsahu šablony.

\begin{figure}[h]
		\centering \tiny \fontfamily{lmtt}\fontseries{l}\selectfont
%		\def\svgwidth{0.5\columnwidth}
		\def\scgscale{1}
		\input{svg/lifecycle2.pdf_tex}
		\normalsize \sffamily
		\captionsetup{width=\linewidth}
		\caption{Životní cyklus presenteru}
		\label{fig:zivotniCyklusPresenteru}
\end{figure}
%\svgobr{Životní cyklus presenteru}{fig:zivotniCyklusPresenteru}{1}{lifecycle2}

Požadavek v Nette je tvořen kombinací \texttt{Presenter:Action}. \it{Signál} rozšiřuje základní požadavek a volá se vždy současně s aktuálním presenterem a akcí. Nejčastěji se signály využívají pro AJAXové požadavky nebo v komponentách, což jsou samostatné znovupoužitelné objekty. Samotný presenter je potomkem komponenty, z toho vyplývá, že komponenty mohou s View komunikovat napřímo pouze díky signálům. V obrázku nejsou uvedeny metody \phpinline{beforeRender()} a \phpinline{afterRender()}, které společně se \phpinline{startup()} a \phpinline{shutdown()} nemají vazbu na \it{akci} nebo \it{signál} a mohou být v~rámci Presenteru definovány právě jednou, slouží k definici společného chování napříč různými akcemi.
\koment{https://doc.nette.org/cs/3.1/presenters#toc-action-action}

\n{3}{Routování} \label{section:routovani}
Veškeré HTTP požadavky klienta míří na soubor \tt{www/index.php}, zde se inicialzuje prostředí Nette, požadavek se přeloží do objektu \phpinline{Nette\Application\Request} a vyvolá se příslušný Presenter. Proces překládání HTTP požadavků, resp. překlad URL se běžně u PHP frameworků označuje jako routování. Router umí URL adresu nejen rozložit ale také složit (neboli vytvářet odkazy). Maska routy říká routeru, jak přeložit URL adresu na tvar \texttt{Presenter:Action}, případně složitější tvary \koment{NetteDocs:https://doc.nette.org/cs/3.1/routing}. Základní instalace Nette obsahuje jednu jedinou routu, která je vidět na výňatku kódu \ref{php:simpleRoute}, tato routa převede URL tvaru \phpinline{https://domain.com/presenter/action} na požadavek tvaru \texttt{Presenter:Action}, přičemž parametr \phpinline{id} se předává jako argument metodá action a render příslušného presenteru. Pokud není v URL přítomná část presenter nebo action, doplní se o výchozí nastavení, zde \texttt{Homepage:default}. Tabulka \ref{tab:simpleRoute} obsahuje příklady překladů pomocí této základní routy.

\begin{listing}[ht]
\phpfile{tex/code/simpleRoute.php}
\caption{Základní routa v Nette}
\label{php:simpleRoute}
\end{listing}

% \tab{popisek}{label}{rozměr (0.0 - 1.0)}{definice sloupců}{obsah} 
\tab{Příklady routování}{tab:simpleRoute}{1}{ll}{
\hline
URL adresa & Nette požadavek \\
\hline
https://domain.com/ & Homepage:default \\
https://domain.com/homepage/about & Homepage:about \\
https://domain.com/article/view/12 & Article:view, id = 12 \\
}

\n{4}{Latte}
Šablonovací systém od vývojáře Nette, představuje výborné spojení s Nette. ...
