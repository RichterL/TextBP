Pod pojmem elektronických voleb jsou zmiňovány převážně dvě odlišná pojetí. Prvním je užití hlasovacích zařízení k uskutečnění volby přímo ve volební místnosti. Neslavně se tento způsob využíval v Nizozemí \cite{GoldsmithCaseSR} a testoval v dalších zemích. U hlasovacích zařízení byla často objevena řada nedostatků a zranitelností a jejich zavedení provázejí protesty a nedůvěra voličů v celý proces voleb \cite{Valasek2020}. I v případě, že jsou zařízení využita pouze částí elektorátu, jakékoli pochybnosti ovlivňují výsledky celku.

Druhým pojetím je tzv. \textit{remote-voting}, neboli volba na dálku a představuje účast ve~volbách prostřednictvím zařízení  pro dálkovou komunikaci, např. přes internet. S~tímto způsobem hlasování bývá spojena údajná výhoda ve vyšší volební účasti, různé studie voleb v Estonsku \cite{estoniaTurnout} nebo Švýcarsku \cite{swissTurnout} nicméně neukazují žádný nebo minimální nárůst účasti. Pro úzce zaměřené hlasování například na univerzitách tento vliv může být podstatně vyšší.

Návrh elektronického volebního schématu nebo protokolu a implementace takového systému je zjevně náročná a je již přes 40 let předmětem výzkumů. Výrazného rozšíření se systémy pro elektronické volby na státních úrovních nicméně nedočkaly a zůstávají na úrovni univerzit, případně lokálních voleb.

\n{2}{Požadavky na elektronický volební systém}\label{section:pozadavky}
Existuje vícero protokolů různé komplexity a s různými cíly. Z existence mnoha dostupných návrhů a implementací elektronických voleb lze vyvodit závěr, že zatím není žádný univerzální protokol nebo standard pro elektronické volby.  Je zřejmé, že protokol elektronických voleb na celostátní úrovni (parlamentní volby apod.) by měl klást podstatně striktnější nároky na takový systém v porovnání s hlasováním (semestrální dotazníky aj.)  na akademické půdě. Chybějící univerzální protokol zároveň nedává jinou možnost než definovat vlastní požadavky pro každý systém zvlášť.
\clearpage
Existující systémy, návrhy i protokoly se, ať už více či méně, shodují v několika základních bodech, které by měl takový systém splňovat. Takto je definoval Schneier~\cite{Schneier1996} v knize \textit{Applied Cryptography}\footnote{
1. Only authorized voters can vote.
2. No one can vote more than once.
3. No one can determine for whom anyone else voted.
4. No one can duplicate anyone else’s vote. (This turns out to be the
hardest requirement.)
5. No one can change anyone else’s vote without being discovered.
6. Every voter can make sure that his vote has been taken into account
in the final tabulation.
Additionally, some voting schemes may have the following requirement:
7. Everyone knows who voted and who didn’t.\cite{Schneier1996}
}:
\begin{enumerate}
	\item Pouze oprávnění voliči mohou volit.
	\item Nikdo nemůže volit více než jednou.
	\item Nikdo nemůže určit, jak volil kdokoli jiný.
	\item Nikdo nemůže duplikovat hlas kohokoli jiného.
	\item Nikdo nemůže pozměnit hlas někoho jiného aniž by byl odhalen.
	\item Každý volič se může přesvědčit, že jejich hlas byl zahrnut v celkovém součtu.
	\item Všichni vědí, kdo volil a kdo ne (nemusí platit pro všechny systémy).
\end{enumerate}


Dalším často zmiňovaným požadavkem na volební systém je jeho \textbf{bezchybnost} (přesnost). Systém je možné považovat za bezchybný, pokud všechny platné hlasy budou zahrnuty ve výsledku a žádný neplatný hlas zahrnut nebude \cite{4285237}\cite{QADAH2007376}\cite{10.1007/978-3-642-03315-5_13}.

Náročným bodem na implementaci je i verifikovatelnost voleb (body 6 a 7). Obecně se rozlišuje univerzální a individuální \cite{4285237}\cite{10.1007/978-3-642-03315-5_13}. Individuální verifikovatelnost představuje možnost voliče ověřit, že jeho hlas byl započítán v celkovém výsledku a že odpovídá tomu, jak hlasoval. Univerzální verifikovatelnost umožňuje komukoli ověřit, že ve volbách jsou započítány hlasy pouze oprávněných voličů.

Tento bod je kontroverzní především tím, že jeho implementace spočívá v poskytnutí nějakého potvrzení o volbě voliči. Toto potvrzení ale může být velice snadno zneužito k manipulaci s výsledky voleb formou skupování hlasů. Některá navrhovaná schémata a systémy se snaží řešit i tuto výzvu \cite{receiptFree}. Dalším podobným problémem voleb na dálku je nemožnost odhalit, pokud někdo volí nesvobodně či pod nátlakem (např. \textit{family voting}\footnote{situace, kdy členové rodiny nevolí samostatně, ale společně - ať už dobrovolně nebo pod nátlakem}).