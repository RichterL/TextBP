Nette patří do skupiny architektonických vzorů známých jako MVC (Model View Controller), přesněji MVP (Model View Presenter). Jako první popsal MVC v roce 1979 Trygve Reenskaug pro programovací jazyk Smalltalk \cite{FowlerMVC}. Základním principem je rozdělení systému do tří samostatných částí - data jako Model a vstup a výstup jako Controller resp. View \cite{Vrana2013}. S vývojem počítačů ustupovala potřeba tohoto dělení, jelikož jedna komponenta systému již uměla obsloužit vstup i výstup zároveň. S~příchodem a~rozmachem internetu se MVC vrátilo a zatím zůstává \cite{zdrojakMVC}.

V kontextu webové aplikace chápeme Model jako data a jejich obsluhu, View jako zobrazení těchto dat uživateli a Controller zpracovává uživatelské vstupy, manipuluje s~Modelem a aktivuje View. Uživatelské rozhraní je v tomto podání tedy kombinací View a Controlleru. Současné frameworky nejčastěji kombinují vzory Front Controller (obsluha HTTP požadavku) a Page Controller (samotná logika konrétní části aplikace)~\cite{FowlerMVC}.

Variantu MVP (Model View Presenter) v současném podání popisuje Fowler \cite{FowlerPassiveView} jako vzor Passive View. Dochází k těsnější vazbě Controlleru (resp. Presenteru) a~View a~zároveň je Model izolován od View. Například v Nette neexistuje obdoba Front Controlleru, už z URL adresy totiž aplikace pozná, který Presenter i jeho metoda je volána. Logika Front Controlleru se tedy rozpustila mezi View a Page Controller, kterému se říká Presenter.