Při návrhu aplikace bylo využito zásad Domain Driven Desingu (dále též DDD), který uvedl Eric Evans ve své knize \textit{Domain-driven Design: Tackling Complexity in~the Heart of Software} \cite{EvansDDD}. Tento styl vývoje software si klade za cíl řešit návrh komplexního řešení pomocí modelu obchodní domény. Úzká polupráce klienta (\textit{doménoví experti}) a vývojářů (\textit{techničtí experti}) probíhá za pomoci společného a~jednotného jazyka (\textit{Ubiquitous Language}). Aplikace by měla být rozdělena do~několika základních vrstev a to konkrétně \cite{EvansDDD}:

\begin{itemize}
	\item \textbf{Prezentační vrstva} - přenáší informace uživateli a obsluhuje jeho požadavky.
	\item \textbf{Aplikační vrstva} - koordinuje práci ostatních objektů, neobsahuje business logiku.
	\item \textbf{Doménová vrstva} - hlavní část DDD, která obsahuje doménové objekty a~kompletní business logiku.
	\item \textbf{Vrstva infrakstruktury} - poskytuje prostředky ostatním vrstvám (komunikace, perzistence aj.).
\end{itemize}
\clearpage
Při srovnání koncepcí vrstev podle MVC/P a DDD lze říci, že se vhodně překrývají, pokud je zajištěno, že Controller neobsauje logiku doménové vrstvy. V~případě MVP pak částečně dochází k prolínání aplikační a prezentační vrstvy, což podle Evanse nehraje zásadní roli, tou je separace doménové vrstvy \cite{EvansDDD}. Dále Evans definuje základními stavební bloky doménové vrstvy následující \cite{EvansDDD}:
\begin{itemize}
	\item \textbf{Value Object} - neměnný (immutable) objekt, který reprezentuje nějakou hodnotu / vlastnost. Může to být telefonní číslo, ale i poštovní adresa skládající se z několika částí (členských proměnných).
	\item \textbf{Entity} - základní objekt, který je jednoznačně identifikován svojí \it{identitou}, jeho vlastnosti mohou být reprezentovány value objekty nebo dalšími entitami, z čehož vznikají agregáty.

	\item \textbf{Aggregate} - skupina entit, která v doméně představuje celek. \it{Root Aggregate} pak představuje vstupní bod pro okolní objekty k agregátovi. Objekty uvnitř agregátu mohou mít libovolné vazby mezi sebou, ale ne na okolní objekty. Převážná část business logiky by se měla odehrávat právě zde.
	\item \textbf{Module} - logické propojení (v kontextu domény) entit a agregátů vytváří moduly
	\item \textbf{Factory} - továrny na objekty zapouzdřují proces vytváření nových objektů. Může~to být metoda agregátu, která vytváří instance jednotlivých entit a value objektů nebo samostatná třída. Využívá se některého z návrhových vzorů Factory \cite{Vrana2013}\cite{Boehmer2015}
	\item \textbf{Service} - pokud nějaká operace nedává smysl v rámci jednoho objektu, může být zapouzdřena do samostatného objektu služby.
	\item \textbf{Repository} - získávání objektů, jejich perzistence (ukládání, mazání) je zapouzdřeno do samostatných repozitářů. 
\end{itemize}