Šablony (\textit{templates}) jsou součástí zobrazovací vrstvy (View) aplikace a jejich účelem je prezentovat uživateli data z aplikační a doménové vrstvy v lidské podobě. Pro jednoduchost popisu byly do této části zařazeny i kaskádové styly (CSS) a JavaScript (JS), přestože nejsou ve striktním podání šablonami.

Jak již bylo popsáno, šablona je těsně navázána na presenter a jeho akci. I pokud presenter nemá definované žádné metody akce (\texttt{action}, \texttt{render}), pokud existuje šablona, může být zobrazena. Každá šablona musí být umístěna v adresáři, který se shoduje s názvem presenteru a název souboru šablony musí odpovídat názvu akce. Prázdná (relativní) URL adresa webu směřuje na HomepagePresenter a jeho akci default. Pokud má existovat i výstup pro danou adresu, šablona \texttt{default.latte} bude umístěna do adresáře \texttt{templates/Homepage/}, přičemž adresář \texttt{templates} je na stejné úrovni jako adresář obsahující presentery.

Výstupem šablony je nejčastěji HTML kód zobrazený v prohlížeči. Nette umožňuje vykreslovat i samostatné části šablon zvané \textit{snippety} \cite{NetteDocs}, čehož je hojně využíváno pro AJAXové požadavky. Presenter po zpracování AJAX požadavku může jako odpověd poslat pouze část šablony ve formě JSON řetězce, který zpracuje JavaScriptová knihovna Naja a vloží na konkrétní místo v HTML dokumentu, který má prohlížeč již načtený. Tímto se podstatně snižuje objem přenesených dat, ale především rychlost načtení požadovaného obsahu. Na uživatele aplikace zároveň působí svižně, jelikož nedochází k překreslování celého obsahu prohlížečem, změní se pouze ta část dokumentu, která je definovná jako snippet.

Snippety jsou v aplikaci použity pro formuláře, datagridy a další. Detail konkrétních voleb obsahuje jednotlivé záložky, které také využívají snippetů a~AJAXových požadavků. Knihovna Naja navíc umí simulovat historii prohlížení i~změnou URL adresy v adresním řádku prohlížeče, zároveň funguje i možnost v~historii listovat pomocí příkazů \textit{zpět} a \textit{vpřed} prohlížeče.

\n{3}{CSS styly}
Aplikace využívá CSS frameworku Bootstrap původně vyvinutý ve společnosti Twitter a v současnosti jeden z nejpoužívanějších CSS frameworků vůbec \cite{Bootstrap}. Bootstrap umožňuje vytvářet responzivní stránky velice snadno pouze použitím CSS tříd. Použitá verze (Bootstrap v4.6) využívá \textit{flexbox} k dosažení responzivního vzhledu. Tento framework (a verze) byl zvolen vzhledem k dostupným rozšířením pro Nette formuláře a datagridy, které tak působí jednotným vzhledem. Některé formuláře musely být i tak vykresleny manuálně, aby bylo dosaženo požadovaného vzhledu, především kvůli nedokonalému zobrazení chyb ve formuláři.

Vlastní a upravené kaskádové styly jsou uloženy v \texttt{custom.css}, další používané CSS soubory jsou závislostmi používaných balíčků. V backendové části jsou také použity ikony FontAwesome \cite{FontAwesome}.

\n{3}{Skripty a balíčky}

Aplikace také využívá několik JavaScriptových knihoven a vlastních skriptů. Vlastními skripty jsou rozšíření pro knihovnu Naja a skript pro šifrování hlasů na straně klienta. Naja byla rozšířena o možnost nuceného přesměrování, indikaci načítání stránky při AJAXovém požadavku, zobrazení modálních oken a uložení obsahu editoru tinyMCE před odesláním formuláře na server.
\begin{itemize}
	\item Naja - obsluha a zpracování AJAX požadavků
	\item jQuery - knihovna pro manipulaci s Document Object Model (DOM)
	\item netteForms - validace formulářů
	\item Bootstrap - knihovna Bootstrap frameworku
	\item toastr.js - zobrazení krátkých stavových zpráv (toastů)
	\item charts.js - interaktivní grafy
	\item tinyMCE - WYSIWYG textový editor ve formulářových polích \texttt{textarea}
	\item SweetAlert2 - uživatelsky přívětivé upozornění a dialogová okna
\end{itemize}
